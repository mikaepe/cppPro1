\documentclass[a4paper,10pt]{article}

\usepackage{amsmath,framed}
\usepackage[latin1]{inputenc} 	

\renewcommand{\d}{\text{d}}
\newcommand{\e}{\text{e}}
\newcommand{\ve}{\mathbf}
\newcommand\numb{\addtocounter{equation}{1}\tag{\theequation}}

\setlength\fboxsep{1.2mm}
\setlength\fboxrule{0.5mm}

 %%Margins	%%%	%%%	%%%	%%%	%%%	%%%	%%%
\usepackage{geometry}
\geometry{
  a4paper,
  left=40mm,
  right=40mm,
  top=40mm,
  bottom=40mm,
}

%%Header & Footer	%%%	%%%	%%%	%%%	%%%	%%%
\usepackage{fancyhdr}
\pagestyle{fancy}
\renewcommand{\headrulewidth}{1pt}
\rhead{Hanna Hultin, Mikael Perssson}
\lhead{P1 SF565}

\title{Project 1, SF2565}
\author{Hanna Hultin, TTMAM2 \\ Mikael Persson, TTMAM2}

\begin{document}
\maketitle

\subsubsection*{Task 1}
Consider the $2N$ degree Taylor polynomial for  $\cos x$
\begin{align*}
  \cos x &\approx p(x) = \sum_{n=0}^{N} (-1)^n \frac{x^{2n}}{(2n)!} \\ 
  \quad &= 1 + (-1) \frac{x^2}{2!} + (-1)^2 \frac{x^4}{4!} + \dots + (-1)^N \frac{x^{2N}}{(2N)!}
  \\
  &= 1 - \frac{x\cdot x}{2\cdot 1} \Big(1 - \frac{x \cdot x}{ 4 \cdot 3} \Big( 
  1 - \dots \Big(1- \frac{x \cdot x}{(2N)(2N-1)}\Big)\dots \Big).
\end{align*}
Hence, the polynomial may be evaluated backwords using the following scheme
\begin{align*}
  b_N &= 1-\frac{x \cdot x}{2N(2N-1)} \\
  b_n &= 1-\frac{x\cdot x}{2n(2n-1)}b_{n+1},\quad n = N-1,N-2,\dots,2,1 \\
  b_1 &= p(x).
\end{align*}
This is Horners' algorithm adjusted for the fact that each second term in the polynomial vanishes.
Similarly for $\sin x$ the polynomial may be computed up to degree $2N+1$ by
\begin{align*}
  b_N &= 1-\frac{x \cdot x}{2N(2N+1)} \\
  b_n &= 1-\frac{x\cdot x}{2n(2n+1)}b_{n+1},\quad n = N-1,N-2,\dots,2,1 \\
  b_1 &= x \cdot p(x).
\end{align*}

\end{document}





